\documentclass[lang=cn,newtx,10pt,scheme=chinese]{elegantbook}

\title{\LaTeX{} 模板 elegantbook.cls 使用示例}
\subtitle{示例1 不带封面使用}

\author{kissdata}

\date{2023/1/11}
% \version{1.0}

\extrainfo{模板作者:Ethan Deng}

\setcounter{tocdepth}{3}

\usepackage{array}
\newcommand{\ccr}[1]{\makecell{{\color{#1}\rule{1cm}{1cm}}}}

% 修改标题页的橙色带
\definecolor{customcolor}{RGB}{32,178,170}
\colorlet{coverlinecolor}{customcolor}
\usepackage{cprotect}

\addbibresource[location=local]{reference.bib} % 参考文献

\begin{document}
	
 %	\maketitle % 封面

这是一个教使用 elegantbook 的 库的案例,为什么用它?因为它在 2022 年底 “定型” 了,最终版本 v4.5。你可以在
 \href{https://github.com/ElegantLaTeX/ElegantBook/releases}{github.com/ElegantLaTeX/ElegantBook} 下载。

下载链接里也有官方的写法教程,本文档演示的东西是:去除封面;去除参考;目录前加一页;章节当作附录的代码框架。
平时做知识笔记可以拿它来跑。

~\\

在目录前多出这一页的方法其实很容易,就是把文字内容写在 \lstinline|\frontmatter| 前面(笑 $\sim \sim$


	\frontmatter  % 目录链接
	\tableofcontents % 目录
	\mainmatter % 章节带序号,目录带页码


\chapter{文档代码}
	
	\section{如何获取本文档源码}
	
	代码在 example 目录的文件 test1.tex
	
	\begin{remark}
		还有一个 Makefile 文件。因为用 tex 运行后会产生一堆文件,建议写完后用它清理一下。
		(不会吧,不会 make clean 在 win 上按了为什么没用还不知道吧 $ \cdots$)
	\end{remark}

	\section{写法框架}
	
	下面的代码复制了就能跑,
	提醒一下,第一行的 \{elegantbook\} 就是 cls 文件名,你要把它放到 tex 文件同一个文件夹下。
\begin{lstlisting}
\documentclass[lang=cn,newtx,10pt,scheme=chinese]{elegantbook}

% ... % 自定义无关标签

\setcounter{tocdepth}{3}
	
\usepackage{array}
\newcommand{\ccr}[1]{\makecell{{\color{#1}\rule{1cm}{1cm}}}}

\usepackage{cprotect}
\addbibresource[location=local]{reference.bib} % 参考文献

\begin{document}
	
	balabalabala % 这个例子的重点

	\frontmatter     % 目录链接,不写的话点目录不跳转
	\tableofcontents % 目录
	\mainmatter      % 章节带序号,目录带页码
		
	\chapter{xxx}   % 第一章,
		这里开始写正文 balabala $\cdots$

	\chapter{xxx} 

\appendix	% 后面的章节都是附录

	\chapter{yyy}   % 附录A

\end{document}

\end{lstlisting}


\chapter{软件开发}

	balabala $ \cdots $

	\section{学什么语言}

	C++, Go
	
\appendix  

\chapter{语法说明}

	这里写附录内容,算是废话输出地,一定要记得在这个章节名前面加上 \lstinline|\appendix| 哦!
	
\end{document}